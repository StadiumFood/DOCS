\section*{Заголовок запроса}
\begin{lstlisting}
[{"key":"Content-Type","value":"application/json"}]
\end{lstlisting}
\hfill

\section*{Ответные данные}       
    \begin{table}[htbp]
    \centering
    \begin{tabularx}{\textwidth}{bsm}
    
    	\rowcolor{titleColor}
        \textbf{Параметр} & \textbf {Тип} & \textbf{Описание} \\  
        
        code & int  & Код ответа \\    \rowcolor{codeColor}
        message & string  & Описание кода ответа \\
        data & struct & Данные \\
    \end{tabularx}
\end{table}

В некоторых случаях могут отсутствуют данные (data), но код ответа и его описание все равно передаются.

\section*{Коды ответных данных}
\begin{table}[htbp]
    \centering
    \begin{tabularx}{\textwidth}{bm}
    
    	\rowcolor{titleColor}
    	\textbf{Код} & \textbf{Описание} \\  
        
        200 & Запрос выполнен успешно \\   \rowcolor{codeColor}
        400 & Запрос не удалось обработать из-за синтаксической ошибки \\
        404 & Сервер не нашел ресурсов \\   \rowcolor{codeColor}
        500 & Сервер не смог обработать запрос \\
    \end{tabularx}
\end{table}

\section*{Пояснение общих процедур}
\begin{table}[htbp]
    \centering
    \begin{tabularx}{\textwidth}{bm}
    
    	\rowcolor{titleColor}
    	\textbf{Процедура} & \textbf{Описание} \\  
        
        getById & Получение определенной записи таблицы по идентификатору\\   \rowcolor{codeColor}
        getAll & Получение всех записей таблицы \\
        delete & Удаление определенной записи таблицы по идентификатору \\   \rowcolor{codeColor}
        seve & Сохранение клиента, данные которого передаются в теле запроса \\
        edit & Изменение данных определенного поля записи таблицы по идентификатору \\   \rowcolor{codeColor}
    \end{tabularx}
\end{table}