\documentclass[14pt,a4paper,report]{report}
\usepackage[a4paper, mag=1000, left=2.5cm, right=1cm, top=2cm, bottom=2cm, headsep=0.7cm, footskip=1cm]{geometry}
\usepackage[utf8]{inputenc}
\usepackage[english,russian]{babel}
\usepackage{indentfirst}
\usepackage[dvipsnames]{xcolor}
\usepackage[colorlinks]{hyperref}
\usepackage{listings} 
\usepackage{fancyhdr}
\usepackage{caption}
\usepackage{graphicx}
\hypersetup{
	colorlinks = true,
	linkcolor  = black
}

\usepackage{titlesec}
\titleformat{\chapter}
{\Large\bfseries} % format
{}                % label
{0pt}             % sep
{\huge}           % before-code

\definecolor{codeColor}{RGB}{246, 248, 250}
\definecolor{titleColor}{RGB}{240, 237, 252}

\DeclareCaptionFont{white}{\color{white}} 

% Listing description
\usepackage{listings} 
\DeclareCaptionFormat{listing}{\colorbox{gray}{\parbox{\textwidth}{#1#2#3}}}
\captionsetup[lstlisting]{format=listing,labelfont=white,textfont=white}
\lstset{ 
	% Listing settings
	inputencoding = utf8,			
	extendedchars = \true, 
	keepspaces = true, 			  	 % Поддержка кириллицы и пробелов в комментариях
	language = Java,            	 	 % Язык программирования (для подсветки)
	stepnumber = 1,               	 % Размер шага между двумя номерами строк
	numbersep = 5pt,              	 % Как далеко отстоят номера строк от подсвечиваемого кода
	backgroundcolor = \color{codeColor}, % Цвет фона подсветки - используем \usepackage{color}
	showspaces = false,           	 % Показывать или нет пробелы специальными отступами
	showstringspaces = false,    	 % Показывать или нет пробелы в строках
	showtabs = false,           	 % Показывать или нет табуляцию в строках
	tabsize = 2,                  	 % Размер табуляции по умолчанию равен 2 пробелам
	captionpos = t,             	 % Позиция заголовка вверху [t] или внизу [b] 
	breaklines = true,           	 % Автоматически переносить строки (да\нет)
	breakatwhitespace = false   	 % Переносить строки только если есть пробел
}

\makeatletter
\@addtoreset{chapter}{part}
\makeatother  

\usepackage{tabularx}
\newcolumntype{b}{>{\hsize=0.2\hsize}X}
\newcolumntype{s}{>{\hsize=0.15\hsize}X}
\newcolumntype{m}{>{\hsize=1\hsize}X}
\usepackage{color, colortbl}

\usepackage{etoolbox}
\usepackage{emptypage}

\usepackage{lipsum}
\usepackage{fancyhdr}

%\fancyfoot[R]{Это простой пример верхнего колонтитула}
\let\Oldpart\part
\newcommand{\parttitle}{}
\renewcommand{\part}[1]{\Oldpart{#1}\renewcommand{\parttitle}{#1}}

\fancypagestyle{plain} {
\fancyhead[LO,LE]{\textit{ \hyperref[toc]{К содержанию} }}
\fancyhead[RO, RE]{}%\parttitle
}

%\patchcmd{\part}{plain}{empty}{}{}

\pagestyle{plain}




\begin{document}


\begin{titlepage}
\vspace*{\fill}
    \begin{center}
      \textbf{\Huge RESTful API}

    \end{center}
    \vspace*{\fill}
\end{titlepage}

\setcounter{page}{2}

\def\contentsname{Содержание}
\tableofcontents \label{toc}
%\hyperref[toc]{К содержанию}
\clearpage


\part{Общая информация} 
\section*{Заголовок запроса}
\begin{lstlisting}
[{"key":"Content-Type","value":"application/json"}]
\end{lstlisting}
\hfill

\section*{Ответные данные}       
    \begin{table}[htbp]
    \centering
    \begin{tabularx}{\textwidth}{bsm}
    
    	\rowcolor{titleColor}
        \textbf{Параметр} & \textbf {Тип} & \textbf{Описание} \\  
        
        code & int  & Код ответа \\    \rowcolor{codeColor}
        message & string  & Описание кода ответа \\
        data & struct & Данные \\
    \end{tabularx}
\end{table}

В некоторых случаях могут отсутствуют данные (data), но код ответа и его описание все равно передаются.

\section*{Коды ответных данных}
\begin{table}[htbp]
    \centering
    \begin{tabularx}{\textwidth}{bm}
    
    	\rowcolor{titleColor}
    	\textbf{Код} & \textbf{Описание} \\  
        
        200 & Запрос выполнен успешно \\   \rowcolor{codeColor}
        400 & Запрос не удалось обработать из-за синтаксической ошибки \\
        404 & Сервер не нашел ресурсов \\   \rowcolor{codeColor}
        500 & Сервер не смог обработать запрос \\
    \end{tabularx}
\end{table}

\section*{Пояснение общих процедур}
\begin{table}[htbp]
    \centering
    \begin{tabularx}{\textwidth}{bm}
    
    	\rowcolor{titleColor}
    	\textbf{Процедура} & \textbf{Описание} \\  
        
        getById & Получение определенной записи таблицы по идентификатору\\   \rowcolor{codeColor}
        getAll & Получение всех записей таблицы \\
        delete & Удаление определенной записи таблицы по идентификатору \\   \rowcolor{codeColor}
        seve & Сохранение клиента, данные которого передаются в теле запроса \\
        edit & Изменение данных определенного поля записи таблицы по идентификатору \\   \rowcolor{codeColor}
    \end{tabularx}
\end{table}



\part{Таблица Client}
%**********************************************************************************************************************%
%**********************************************************************************************************************%
%********************************************************Client********************************************************%
%**********************************************************************************************************************%
%**********************************************************************************************************************%


%********************************************************getById********************************************************%
\chapter{getById}

\section*{Метод}
GET

\section*{Структура запроса}
\begin{lstlisting}
localhost:9999/client/{userId}
\end{lstlisting}
\hfill

\section*{Параметры запроса}
\begin{table}[htbp]
    \centering
    \begin{tabularx}{\textwidth}{bsm}
    
    	\rowcolor{titleColor}
        \textbf{Параметр} & \textbf {Тип} & \textbf{Описание} \\  
        
         userId & int  & Id клиента \\
    \end{tabularx}
\end{table}

\section*{Ответные данные}

\begin{table}[htbp]
    \centering
    \begin{tabularx}{\textwidth}{bsm}
    
    	\rowcolor{titleColor}
        \textbf{Параметр} & \textbf {Тип} & \textbf{Описание} \\  
        

		id & int  & Id клиента \\   \rowcolor{codeColor}
        name & string  & Имя \\   
        surName & string  & Фамилия \\ \rowcolor{codeColor}
        oldName & string  &  Отчество \\   
        email & string  & Электронный адрес \\ \rowcolor{codeColor}
        regDate & datetime  & Дата регистрации учетной записи \\ 
        remDate & datetime  & Дата удаления учетной записи  \\ \rowcolor{codeColor}
        idValidate & bool  & Признак подтверждения учетной записи \\
    \end{tabularx}
\end{table}

\section*{Пример}

\subsection*{Запрос}

\begin{lstlisting}
localhost:9999/client/2
\end{lstlisting}
\hfill

\subsection*{Ответ}

\begin{lstlisting}
{
    "code": 200,
    "message": "OK",
    "data": {
        "surName": null,
        "oldName": null,
        "name": "denis",
        "regDate": null,
        "remDate": null,
        "id": 2,
        "email": "myEmail"
    }
}
\end{lstlisting}
\hfill


%********************************************************getAll********************************************************%
\chapter{getAll}

\section*{Метод}
GET

\section*{Структура запроса}
\begin{lstlisting}
localhost:9999/allClients
\end{lstlisting}
\hfill

\section*{Параметры запроса}
Отсутствуют.

\section*{Ответные данные}

\begin{table}[htbp]
    \centering
    \begin{tabularx}{\textwidth}{bsm}
    
    	\rowcolor{titleColor}
        \textbf{Параметр} & \textbf {Тип} & \textbf{Описание} \\  
        

        data & JSONArray  & Список клиентов \\   \rowcolor{codeColor}

    \end{tabularx}
\end{table}

\section*{Пример}

\subsection*{Запрос}

\begin{lstlisting}
localhost:9999/allClients
\end{lstlisting}
\hfill

\subsection*{Ответ}

\begin{lstlisting}
{
	"code" : 200,
	"message" : "OK" ,
	"data": [{
		"surName": "someSurName",
		"oldName": "someOldName",
		"name": "someName",
		"regDate ": /*Позже уточним*/,
		"remDate": /*Позже уточним*/,
		"id": 1 ,
		"email ": "myEmail"
	},
	{
		"surName": "someSurName",
		"oldName": "someOldName",
		"name": "someName",
		"regDate ": /*Позже уточним*/,
		"remDate": /*Позже уточним*/,
		"id": 2 ,
		"email ": "myEmail"
	}
	]
}
\end{lstlisting}
\hfill


%********************************************************delete********************************************************%
\chapter{delete}

\section*{Метод}
DELETE

\section*{Структура запроса}
\begin{lstlisting}
localhost:9999/client/{userId}
\end{lstlisting}
\hfill

\section*{Параметры запроса}
\begin{table}[htbp]
    \centering
    \begin{tabularx}{\textwidth}{bsm}
    
    	\rowcolor{titleColor}
        \textbf{Параметр} & \textbf {Тип} & \textbf{Описание} \\  
        
         userId & int  & Id клиента \\
    \end{tabularx}
\end{table}

\section*{Ответные данные}
Отсутствуют.

\section*{Пример}

\subsection*{Запрос}

\begin{lstlisting}
localhost:9999/client/2
\end{lstlisting}
\hfill

\subsection*{Ответ}

\begin{lstlisting}
{
	"code": 200,
	"message": "OK"
}
\end{lstlisting}
\hfill


%********************************************************save********************************************************%
\chapter{save}

\section*{Метод}
POST

\section*{Структура запроса}
\begin{lstlisting}
localhost:9999/client
\end{lstlisting}
\hfill

\section*{Параметры запроса}
\begin{table}[htbp]
    \centering
    \begin{tabularx}{\textwidth}{bsm}
    
    	\rowcolor{titleColor}
        \textbf{Параметр} & \textbf {Тип} & \textbf{Описание} \\  
        
        surName & string  & Фамилия \\ 
        oldName & string  &  Отчество \\ \rowcolor{codeColor}
        name & string  & Имя \\   
        email & string  & Электронный адрес \\ \rowcolor{codeColor}
        password & string & Пароль \\
    \end{tabularx}
\end{table}

\section*{Ответные данные}
Отсутствуют.

\section*{Пример}

\subsection*{Запрос}

\begin{lstlisting}
localhost:9999/client
{
	"surName": "someSurName",
	"oldName": "someOldName",
	"name": "someName",
	"regDate ": /*Позже уточним*/,
	"remDate": /*Позже уточним*/,
	"email ": "myEmail",
	"password": "pass"
}
\end{lstlisting}
\hfill

\subsection*{Ответ}

\begin{lstlisting}
{
	"code": 200,
	"message": "OK"
}
\end{lstlisting}
\hfill


%********************************************************edit********************************************************%
\chapter{edit}

\section*{Метод}
PATCH

\section*{Структура запроса}
\begin{lstlisting}
localhost:9999/client
\end{lstlisting}
\hfill

\section*{Параметры запроса}
\begin{table}[htbp]
    \centering
    \begin{tabularx}{\textwidth}{bscm}
    
    	\rowcolor{titleColor}
        \textbf{Параметр} & \textbf {Тип} & \textbf {Обязательно} & \textbf{Описание} \\  
        
        id & int  & Да & Id клиента \\ \rowcolor{codeColor}
        name & string & Нет & Имя \\   
        surName & string & Нет & Фамилия \\ \rowcolor{codeColor}
        oldName & string  & Нет &  Отчество \\   
        email & string & Нет & Электронный адрес \\ \rowcolor{codeColor}
        regDate & datetime & Нет & Дата регистрации учетной записи \\ 
        remDate & datetime & Нет & Дата удаления учетной записи  \\ \rowcolor{codeColor}
        password & string & Нет & Пароль \\
        idValidate & bool & Нет & Признак подтверждения учетной записи \\ \rowcolor{codeColor}
    \end{tabularx}
\end{table}

\section*{Ответные данные}
Отсутствуют.

\section*{Пример}

\subsection*{Запрос}

\begin{lstlisting}
localhost:9999/client
{
	"id": 2,
	"surName": "otherSurName"
}
\end{lstlisting}
\hfill

\subsection*{Ответ}

\begin{lstlisting}
{
	"code": 200,
	"message": "OK"
}
\end{lstlisting}
\hfill



\part{Таблица Position}
%************************************************************************************************************************%
%************************************************************************************************************************%
%********************************************************Position********************************************************%
%************************************************************************************************************************%
%************************************************************************************************************************%


%********************************************************getById********************************************************%
\chapter{getById}

\section*{Метод}
GET

\section*{Структура запроса}
\begin{lstlisting}
localhost:9999/position/{positionId}
\end{lstlisting}
\hfill

\section*{Параметры запроса}
\begin{table}[htbp]
    \centering
    \begin{tabularx}{\textwidth}{bsm}
    
    	\rowcolor{titleColor}
        \textbf{Параметр} & \textbf{Тип} & \textbf{Описание} \\  
        
         positionId & int  & Id должности \\
    \end{tabularx}
\end{table}

\section*{Ответные данные}
\begin{table}[htbp]
    \centering
    \begin{tabularx}{\textwidth}{bsm}
    
    	\rowcolor{titleColor}
        \textbf{Параметр} & \textbf{Тип} & \textbf{Описание} \\  
        

		id & int  & Id должности \\   \rowcolor{codeColor}
        name & string  &  Наименование должности \\   

    \end{tabularx}
\end{table}

\section*{Пример}

\subsection*{Запрос}

\begin{lstlisting}
localhost:9999/product/2
\end{lstlisting}
\hfill

\subsection*{Ответ}

\begin{lstlisting}
{
    "code": 200,
    "message": "OK",
    "data": {
        "id": 2,
        "typeId": 1,
        "name": "product name",
        "decription": "product description",
        "price": 100
    }
}
\end{lstlisting}
\hfill



\part{Таблица Pasport}
%************************************************************************************************************************%
%************************************************************************************************************************%
%********************************************************Pasport********************************************************%
%************************************************************************************************************************%
%************************************************************************************************************************%


%********************************************************getById********************************************************%
\chapter{getById}

\section*{Метод}
GET

\section*{Структура запроса}
\begin{lstlisting}
localhost:9999/pasport/{pasportId}
\end{lstlisting}
\hfill

\section*{Параметры запроса}
\begin{table}[htbp]
    \centering
    \begin{tabularx}{\textwidth}{bsm}
    
    	\rowcolor{titleColor}
        \textbf{Параметр} & \textbf{Тип} & \textbf{Описание} \\  
        
         pasportId & int  & Id паспортных данных сотрудника \\
    \end{tabularx}
\end{table}

\section*{Ответные данные}
\begin{table}[htbp]
    \centering
    \begin{tabularx}{\textwidth}{bsm}
    
    	\rowcolor{titleColor}
        \textbf{Параметр} & \textbf{Тип} & \textbf{Описание} \\  
        
		id & int  & Id паспортных данных сотрудника \\   \rowcolor{codeColor}
        series & int  &  Серия паспорта\\   
        number & int  & Номер паспорта \\ \rowcolor{codeColor}
        birthday & date  &  Дата рождения \\   
        sex & char  & Пол \\ \rowcolor{codeColor}
        address & string  & Адрес регистрации \\ 
        editionDate & date  &  Дата выдачи \\ \rowcolor{codeColor}
        editionPlace & string  &  Орган, выдавший паспорт \\   
    \end{tabularx}
\end{table}

\section*{Пример}

\subsection*{Запрос}

\begin{lstlisting}
localhost:9999/pasport/2
\end{lstlisting}
\hfill

\subsection*{Ответ}

\begin{lstlisting}
{
    "code": 200,
    "message": "OK",
    "data": {
        "id": 2,
        "series": 1111,
        "number": 222222,
        "birthday": 95.12.12,
        "sex": m,
        "address": ""ул. Улица, д.1, кв. 1",
        "editionDate": 15.01.01,
        "editionPlace": "ТП №1 ...",
    }
}
\end{lstlisting}
\hfill


%********************************************************getAll********************************************************%
\chapter{getAll}

\section*{Метод}
GET

\section*{Структура запроса}
\begin{lstlisting}
localhost:9999/allPasports
\end{lstlisting}
\hfill

\section*{Параметры запроса}
Отсутствуют.

\section*{Ответные данные}

\begin{table}[htbp]
    \centering
    \begin{tabularx}{\textwidth}{bsm}
    
    	\rowcolor{titleColor}
        \textbf{Параметр} & \textbf {Тип} & \textbf{Описание} \\  
        

        data & JSONArray  & Список паспортных данных \\   \rowcolor{codeColor}

    \end{tabularx}
\end{table}

\section*{Пример}

\subsection*{Запрос}

\begin{lstlisting}
localhost:9999/allPasports
\end{lstlisting}
\hfill

\subsection*{Ответ}

\begin{lstlisting}
{
	"code" : 200,
	"message" : "OK" ,
	"data": [{
        "id": 2,
        "series": 1111,
        "number": 222222,
        "birthday": 95.12.12,
        "sex": m,
        "address": ""ул. Улица, д.1, кв. 1",
        "editionDate": 15.01.01,
        "editionPlace": "ТП №1 ...",
	},
	{
        "id": 2,
        "series": 1112,
        "number": 222221,
        "birthday": 95.12.12,
        "sex": m,
        "address": ""ул. Улица, д.1, кв. 2",
        "editionDate": 15.01.01,
        "editionPlace": "ТП №1 ...",
	}
	]
}
\end{lstlisting}
\hfill


%********************************************************delete********************************************************%
\chapter{delete}

\section*{Метод}
DELETE

\section*{Структура запроса}
\begin{lstlisting}
localhost:9999/pasport/{pasportId}
\end{lstlisting}
\hfill

\section*{Параметры запроса}
\begin{table}[htbp]
    \centering
    \begin{tabularx}{\textwidth}{bsm}
    
    	\rowcolor{titleColor}
        \textbf{Параметр} & \textbf {Тип} & \textbf{Описание} \\  
        
         pasportId & int  & Id паспортных данных \\
    \end{tabularx}
\end{table}

\section*{Ответные данные}
Отсутствуют.

\section*{Пример}

\subsection*{Запрос}

\begin{lstlisting}
localhost:9999/pasport/1
\end{lstlisting}
\hfill

\subsection*{Ответ}

\begin{lstlisting}
{
	"code": 200,
	"message": "OK"
}
\end{lstlisting}
\hfill


%********************************************************save********************************************************%
\chapter{save}

\section*{Метод}
POST

\section*{Структура запроса}
\begin{lstlisting}
localhost:9999/pasport
\end{lstlisting}
\hfill

\section*{Параметры запроса}
\begin{table}[htbp]
    \centering
    \begin{tabularx}{\textwidth}{bsm}
    
    	\rowcolor{titleColor}
        \textbf{Параметр} & \textbf {Тип} & \textbf{Описание} \\  
        
        series & int  &  Серия паспорта\\   \rowcolor{codeColor}
        number & int  & Номер паспорта \\ 
        birthday & date  &  Дата рождения \\   \rowcolor{codeColor}
        sex & char  & Пол \\ 
        address & string  & Адрес регистрации \\ \rowcolor{codeColor}
        editionDate & date  &  Дата выдачи \\ 
        editionPlace & string  &  Орган, выдавший паспорт \\   \rowcolor{codeColor}
    \end{tabularx}
\end{table}

\section*{Ответные данные}
Отсутствуют.

\section*{Пример}

\subsection*{Запрос}

\begin{lstlisting}
localhost:9999/client
{
	"series": 1113,
    "number": 222223,
    "birthday": 95.12.12,
    "sex": m,
    "address": ""ул. Улица, д.1, кв. 3",
    "editionDate": 15.01.01,
    "editionPlace": "ТП №1 ...",
}
\end{lstlisting}
\hfill

\subsection*{Ответ}

\begin{lstlisting}
{
	"code": 200,
	"message": "OK"
}
\end{lstlisting}
\hfill


%********************************************************edit********************************************************%
\chapter{edit}

\section*{Метод}
PATCH

\section*{Структура запроса}
\begin{lstlisting}
localhost:9999/pasport
\end{lstlisting}
\hfill

\section*{Параметры запроса}
\begin{table}[htbp]
    \centering
    \begin{tabularx}{\textwidth}{bscm}
    
    	\rowcolor{titleColor}
        \textbf{Параметр} & \textbf {Тип} & \textbf {Обязательно} & \textbf{Описание} \\  
        
        id & int  & Да & Id клиента \\ \rowcolor{codeColor}
        series & int & Нет & Серия паспорта \\   
        number & int & Нет & Номер паспорта \\ \rowcolor{codeColor}
        birthday & date  & Нет &  Дата рождения \\   
        sex & char & Нет & Пол \\ \rowcolor{codeColor}
        address & string & Нет  & Адрес регистрации \\ 
        editionDate & date & Нет  & Дата выдачи  \\ \rowcolor{codeColor}
        editionPlace & string & Нет & Орган, выдавший паспорт \\  
    \end{tabularx}
\end{table}

\section*{Ответные данные}
Отсутствуют.

\section*{Пример}

\subsection*{Запрос}

\begin{lstlisting}
localhost:9999/client
{
	"id": 2,
	"series": 3333
}
\end{lstlisting}
\hfill

\subsection*{Ответ}

\begin{lstlisting}
{
	"code": 200,
	"message": "OK"
}
\end{lstlisting}
\hfill



\part{Таблица SellEntry}



\part{Таблица Product}
%************************************************************************************************************************%
%************************************************************************************************************************%
%********************************************************Product********************************************************%
%************************************************************************************************************************%
%************************************************************************************************************************%


%********************************************************getById********************************************************%
\chapter{getById}

\section*{Метод}
GET

\section*{Структура запроса}
\begin{lstlisting}
localhost:9999/product/{productId}
\end{lstlisting}
\hfill

\section*{Параметры запроса}
\begin{table}[htbp]
    \centering
    \begin{tabularx}{\textwidth}{bsm}
    
    	\rowcolor{titleColor}
        \textbf{Параметр} & \textbf {Тип} & \textbf{Описание} \\  
        
         productId & int  & Id продукта \\
    \end{tabularx}
\end{table}

\section*{Ответные данные}

\begin{table}[htbp]
    \centering
    \begin{tabularx}{\textwidth}{bsm}
    
    	\rowcolor{titleColor}
        \textbf{Параметр} & \textbf {Тип} & \textbf{Описание} \\  
        

		id & int  & Id продукта \\   \rowcolor{codeColor}
        typeId & int  & Id типа продукта \\   
        name & string  & Наименование \\ \rowcolor{codeColor}
        decription & string  &  Описание \\   
        price & float  & Цена \\ \rowcolor{codeColor}
    \end{tabularx}
\end{table}

\section*{Пример}

\subsection*{Запрос}

\begin{lstlisting}
localhost:9999/product/2
\end{lstlisting}
\hfill

\subsection*{Ответ}

\begin{lstlisting}
{
    "code": 200,
    "message": "OK",
    "data": {
        "id": 2,
        "typeId": 1,
        "name": "product name",
        "decription": "product description",
        "price": 100
    }
}
\end{lstlisting}
\hfill


%********************************************************getAll********************************************************%
\chapter{getAll}

\section*{Метод}
GET

\section*{Структура запроса}
\begin{lstlisting}
localhost:9999/allProducts
\end{lstlisting}
\hfill

\section*{Параметры запроса}
Отсутствуют.

\section*{Ответные данные}

\begin{table}[htbp]
    \centering
    \begin{tabularx}{\textwidth}{bsm}
    
    	\rowcolor{titleColor}
        \textbf{Параметр} & \textbf {Тип} & \textbf{Описание} \\  
        

        products & list  & Список продуктов \\   \rowcolor{codeColor}

    \end{tabularx}
\end{table}

\section*{Пример}

\subsection*{Запрос}

\begin{lstlisting}
localhost:9999/allClients
\end{lstlisting}
\hfill

\subsection*{Ответ}

\begin{lstlisting}
{
	"code" : 200,
	"message" : "OK" ,
	"data": [{
        "id": 2,
        "typeId": 1,
        "name": "product name",
        "decription": "product description",
        "price": 100
	},
	{
        "id": 3,
        "typeId": 1,
        "name": "other product name",
        "decription": "other product description",
        "price": 100
	}
	]
}
\end{lstlisting}
\hfill


%********************************************************delete********************************************************%
\chapter{delete}

\section*{Метод}
DELETE

\section*{Структура запроса}
\begin{lstlisting}
localhost:9999/product/{productId}
\end{lstlisting}
\hfill

\section*{Параметры запроса}
\begin{table}[htbp]
    \centering
    \begin{tabularx}{\textwidth}{bsm}
    
    	\rowcolor{titleColor}
        \textbf{Параметр} & \textbf {Тип} & \textbf{Описание} \\  
        
         productId & int  & Id продукта \\
    \end{tabularx}
\end{table}

\section*{Ответные данные}
Отсутствуют.

\section*{Пример}

\subsection*{Запрос}

\begin{lstlisting}
localhost:9999/product/2
\end{lstlisting}
\hfill

\subsection*{Ответ}

\begin{lstlisting}
{
	"code": 200,
	"message": "OK"
}
\end{lstlisting}
\hfill


%********************************************************save********************************************************%
\chapter{save}

\section*{Метод}
POST

\section*{Структура запроса}
\begin{lstlisting}
localhost:9999/product
\end{lstlisting}
\hfill

\section*{Параметры запроса}
\begin{table}[htbp]
    \centering
    \begin{tabularx}{\textwidth}{bsm}
    
    	\rowcolor{titleColor}
        \textbf{Параметр} & \textbf {Тип} & \textbf{Описание} \\  
        
        typeId & int  & Id типа продукта \\   \rowcolor{codeColor}
        name & string  & Наименование \\ 
        decription & string  &  Описание \\   \rowcolor{codeColor}
        price & float  & Цена \\ 
    \end{tabularx}
\end{table}

\section*{Ответные данные}
Отсутствуют.

\section*{Пример}

\subsection*{Запрос}

\begin{lstlisting}
localhost:9999/client
{
        "typeId": 3,
        "name": "other product name",
        "decription": "other product description",
        "price": 100
}
\end{lstlisting}
\hfill

\subsection*{Ответ}

\begin{lstlisting}
{
	"code": 200,
	"message": "OK"
}
\end{lstlisting}
\hfill


%********************************************************edit********************************************************%
\chapter{edit}

\section*{Метод}
PATCH

\section*{Структура запроса}
\begin{lstlisting}
localhost:9999/product
\end{lstlisting}
\hfill

\section*{Параметры запроса}
\begin{table}[htbp]
    \centering
    \begin{tabularx}{\textwidth}{bscm}
    
    	\rowcolor{titleColor}
        \textbf{Параметр} & \textbf {Тип} & \textbf {Обязательно} & \textbf{Описание} \\  
        
        id & int  & Да & Id продукта \\ \rowcolor{codeColor}
        typeId & int & Нет & Id типа продукта \\   
        name & string & Нет & Наименование \\ \rowcolor{codeColor}
        description & string  & Нет &  Описание \\   
        price & float & Нет & Цена \\ \rowcolor{codeColor}
    \end{tabularx}
\end{table}

\section*{Ответные данные}
Отсутствуют.

\section*{Пример}

\subsection*{Запрос}

\begin{lstlisting}
localhost:9999/client
{
	"id": 2,
	"name": "otherProduct"
}
\end{lstlisting}
\hfill

\subsection*{Ответ}

\begin{lstlisting}
{
	"code": 200,
	"message": "OK"
}
\end{lstlisting}
\hfill



\part{Таблица Type}
%*********************************************************************************************************************%
%*********************************************************************************************************************%
%********************************************************Type********************************************************%
%*********************************************************************************************************************%
%*********************************************************************************************************************%


%********************************************************getById********************************************************%
\chapter{getById}

\section*{Метод}
GET

\section*{Структура запроса}
\begin{lstlisting}
localhost:9999/type/{typeId}
\end{lstlisting}
\hfill

\section*{Параметры запроса}
\begin{table}[htbp]
    \centering
    \begin{tabularx}{\textwidth}{bsm}
    
    	\rowcolor{titleColor}
        \textbf{Параметр} & \textbf {Тип} & \textbf{Описание} \\  
        
         typeId & int  & Id типа продукта \\
    \end{tabularx}
\end{table}

\section*{Ответные данные}

\begin{table}[htbp]
    \centering
    \begin{tabularx}{\textwidth}{bsm}
    
    	\rowcolor{titleColor}
        \textbf{Параметр} & \textbf {Тип} & \textbf{Описание} \\  
        

		id & int  & Id типа продукта \\   \rowcolor{codeColor}
        name & string  & Наименование типа продукта \\   
    \end{tabularx}
\end{table}

\section*{Пример}

\subsection*{Запрос}

\begin{lstlisting}
localhost:9999/product/2
\end{lstlisting}
\hfill

\subsection*{Ответ}

\begin{lstlisting}
{
    "code": 200,
    "message": "OK",
    "data": {
        "id": 1,
        "type": "some type"
    }
}
\end{lstlisting}
\hfill


%********************************************************getAll********************************************************%
\chapter{getAll}

\section*{Метод}
GET

\section*{Структура запроса}
\begin{lstlisting}
localhost:9999/allTypes
\end{lstlisting}
\hfill

\section*{Параметры запроса}
Отсутствуют.

\section*{Ответные данные}

\begin{table}[htbp]
    \centering
    \begin{tabularx}{\textwidth}{bsm}
    
    	\rowcolor{titleColor}
        \textbf{Параметр} & \textbf {Тип} & \textbf{Описание} \\  
        

        types & list  & Список типов продуктов \\   \rowcolor{codeColor}

    \end{tabularx}
\end{table}

\section*{Пример}

\subsection*{Запрос}

\begin{lstlisting}
localhost:9999/allClients
\end{lstlisting}
\hfill

\subsection*{Ответ}

\begin{lstlisting}
{
	"code" : 200,
	"message" : "OK" ,
	"data": [{
        "id": 1,
        "type": "some type 1"
	},
	{
        "id": 2,
        "type": "some type 2"
	}
	]
}
\end{lstlisting}
\hfill


%********************************************************delete********************************************************%
\chapter{delete}

\section*{Метод}
DELETE

\section*{Структура запроса}
\begin{lstlisting}
localhost:9999/type/{typeId}
\end{lstlisting}
\hfill

\section*{Параметры запроса}
\begin{table}[htbp]
    \centering
    \begin{tabularx}{\textwidth}{bsm}
    
    	\rowcolor{titleColor}
        \textbf{Параметр} & \textbf {Тип} & \textbf{Описание} \\  
        
         typeId & int  & Id типа продукта \\
    \end{tabularx}
\end{table}

\section*{Ответные данные}
Отсутствуют.

\section*{Пример}

\subsection*{Запрос}

\begin{lstlisting}
localhost:9999/type/2
\end{lstlisting}
\hfill

\subsection*{Ответ}

\begin{lstlisting}
{
	"code": 200,
	"message": "OK"
}
\end{lstlisting}
\hfill


%********************************************************save********************************************************%
\chapter{save}

\section*{Метод}
POST

\section*{Структура запроса}
\begin{lstlisting}
localhost:9999/type
\end{lstlisting}
\hfill

\section*{Параметры запроса}
\begin{table}[htbp]
    \centering
    \begin{tabularx}{\textwidth}{bsm}
    
    	\rowcolor{titleColor}
        \textbf{Параметр} & \textbf {Тип} & \textbf{Описание} \\  
        
        name & string  & Тип продукта \\ 
    \end{tabularx}
\end{table}

\section*{Ответные данные}
Отсутствуют.

\section*{Пример}

\subsection*{Запрос}

\begin{lstlisting}
localhost:9999/client
{
        "name": "some type 3"
}
\end{lstlisting}
\hfill

\subsection*{Ответ}

\begin{lstlisting}
{
	"code": 200,
	"message": "OK"
}
\end{lstlisting}
\hfill


%********************************************************edit********************************************************%
\chapter{edit}

\section*{Метод}
PATCH

\section*{Структура запроса}
\begin{lstlisting}
localhost:9999/type
\end{lstlisting}
\hfill

\section*{Параметры запроса}
\begin{table}[htbp]
    \centering
    \begin{tabularx}{\textwidth}{bscm}
    
    	\rowcolor{titleColor}
        \textbf{Параметр} & \textbf {Тип} & \textbf {Обязательно} & \textbf{Описание} \\  
        
        id & int  & Да & Id клиента \\ \rowcolor{codeColor}
        name & string & Нет & Тип продукта \\   
    \end{tabularx}
\end{table}

\section*{Ответные данные}
Отсутствуют.

\section*{Пример}

\subsection*{Запрос}

\begin{lstlisting}
localhost:9999/client
{
	"id": 2,
	"name": "other_type"
}
\end{lstlisting}
\hfill

\subsection*{Ответ}

\begin{lstlisting}
{
	"code": 200,
	"message": "OK"
}
\end{lstlisting}
\hfill



\part{Таблица Sell}
%********************************************************************************************************************%
%********************************************************************************************************************%
%********************************************************Sell********************************************************%
%********************************************************************************************************************%
%********************************************************************************************************************%


%********************************************************getById********************************************************%
\chapter{getById}

\section*{Метод}
GET

\section*{Структура запроса}
\begin{lstlisting}
localhost:9999/sell/{sellId}
\end{lstlisting}
\hfill

\section*{Параметры запроса}
\begin{table}[htbp]
    \centering
    \begin{tabularx}{\textwidth}{bsm}
    
    	\rowcolor{titleColor}
        \textbf{Параметр} & \textbf{Тип} & \textbf{Описание} \\  
        
         sellId & int  & Id заказа \\
    \end{tabularx}
\end{table}

\section*{Ответные данные}

\begin{table}[htbp]
    \centering
    \begin{tabularx}{\textwidth}{bsm}
    
    	\rowcolor{titleColor}
        \textbf{Параметр} & \textbf{Тип} & \textbf{Описание} \\  
        

		id & int  & Id заказа \\   \rowcolor{codeColor}
        employeeId & int  & Id сотрудника, который закончил заказ \\   
        userId & int  & Id клиента \\ \rowcolor{codeColor}
        isCompleted & bool  &  Признак завершенности заказа \\   
        sellDate & timestamp  & Время выдачи заказа \\ \rowcolor{codeColor}
    \end{tabularx}
\end{table}

\section*{Пример}

\subsection*{Запрос}

\begin{lstlisting}
localhost:9999/sell/2
\end{lstlisting}
\hfill

\subsection*{Ответ}

\begin{lstlisting}
{
    "code": 200,
    "message": "OK",
    "data": {
        "id": 2,
        "employeeId": 1,
        "userId": 2,
        "isCompleted": true,
        "sellDate": 17.02.02
    }
}
\end{lstlisting}
\hfill


%********************************************************getAll********************************************************%
\chapter{getAll}

\section*{Метод}
GET

\section*{Структура запроса}
\begin{lstlisting}
localhost:9999/allSells
\end{lstlisting}
\hfill

\section*{Параметры запроса}
Отсутствуют.

\section*{Ответные данные}

\begin{table}[htbp]
    \centering
    \begin{tabularx}{\textwidth}{bsm}
    
    	\rowcolor{titleColor}
        \textbf{Параметр} & \textbf {Тип} & \textbf{Описание} \\  
        

        data & JSONArray  & Список заказов \\   \rowcolor{codeColor}

    \end{tabularx}
\end{table}

\section*{Пример}

\subsection*{Запрос}

\begin{lstlisting}
localhost:9999/allSells
\end{lstlisting}
\hfill

\subsection*{Ответ}

\begin{lstlisting}
{
	"code" : 200,
	"message" : "OK" ,
	"data": [{
        "id": 2,
        "employeeId": 1,
        "userId": 2,
        "isCompleted": true,
        "sellDate": 17.02.02
	},
	{
        "id": 3,
        "employeeId": 2,
        "userId": 3
        "isCompleted": true,
        "sellDate": 17.02.02
	}
	]
}
\end{lstlisting}
\hfill


%********************************************************delete********************************************************%
\chapter{delete}

\section*{Метод}
DELETE

\section*{Структура запроса}
\begin{lstlisting}
localhost:9999/sell/{sellId}
\end{lstlisting}
\hfill

\section*{Параметры запроса}
\begin{table}[htbp]
    \centering
    \begin{tabularx}{\textwidth}{bsm}
    
    	\rowcolor{titleColor}
        \textbf{Параметр} & \textbf {Тип} & \textbf{Описание} \\  
        
         sellId & int  & Id заказа \\
    \end{tabularx}
\end{table}

\section*{Ответные данные}
Отсутствуют.

\section*{Пример}

\subsection*{Запрос}

\begin{lstlisting}
localhost:9999/sell/2
\end{lstlisting}
\hfill

\subsection*{Ответ}

\begin{lstlisting}
{
	"code": 200,
	"message": "OK"
}
\end{lstlisting}
\hfill


%********************************************************save********************************************************%
\chapter{save}

\section*{Метод}
POST

\section*{Структура запроса}
\begin{lstlisting}
localhost:9999/sell
\end{lstlisting}
\hfill

\section*{Параметры запроса}
\begin{table}[htbp]
    \centering
    \begin{tabularx}{\textwidth}{bsm}
    
    	\rowcolor{titleColor}
        \textbf{Параметр} & \textbf {Тип} & \textbf{Описание} \\  
        
        employeeId & int  & Id сотрудника, который закончил заказ \\   \rowcolor{codeColor}
        userId & int  & Id клиента \\ 
        isCompleted & bool  &  Признак завершенности заказа \\   \rowcolor{codeColor}
        sellDate & timestamp  & Время выдачи заказа \\ 
    \end{tabularx}
\end{table}

\section*{Ответные данные}
Отсутствуют.

\section*{Пример}

\subsection*{Запрос}

\begin{lstlisting}
localhost:9999/client
{
    "employeeId": 2,
    "userId": 3
    "isCompleted": true,
    "sellDate": 17.02.02
}
\end{lstlisting}
\hfill

\subsection*{Ответ}

\begin{lstlisting}
{
	"code": 200,
	"message": "OK"
}
\end{lstlisting}
\hfill


%********************************************************edit********************************************************%
\chapter{edit}

\section*{Метод}
PATCH

\section*{Структура запроса}
\begin{lstlisting}
localhost:9999/sell
\end{lstlisting}
\hfill

\section*{Параметры запроса}
\begin{table}[htbp]
    \centering
    \begin{tabularx}{\textwidth}{bscm}
    
    	\rowcolor{titleColor}
        \textbf{Параметр} & \textbf {Тип} & \textbf {Обязательно} & \textbf{Описание} \\  
        
		id & int & Да & Id заказа \\   \rowcolor{codeColor}
        employeeId & int & Нет & Id сотрудника, который закончил заказ \\   
        userId & int & Нет & Id клиента \\ \rowcolor{codeColor}
        isCompleted & bool & Нет &  Признак завершенности заказа \\   
        sellDate & timestamp & Нет & Время выдачи заказа \\ \rowcolor{codeColor}
    \end{tabularx}
\end{table}

\section*{Ответные данные}
Отсутствуют.

\section*{Пример}

\subsection*{Запрос}

\begin{lstlisting}
localhost:9999/sell
{
	"id": 2,
	"employeeId": 3
}
\end{lstlisting}
\hfill

\subsection*{Ответ}

\begin{lstlisting}
{
	"code": 200,
	"message": "OK"
}
\end{lstlisting}
\hfill



\part{Таблица ProductAvaliability}



\part{Таблица Storage}
%************************************************************************************************************************%
%************************************************************************************************************************%
%********************************************************Storage********************************************************%
%************************************************************************************************************************%
%************************************************************************************************************************%


%********************************************************getById********************************************************%
\chapter{getById}

\section*{Метод}
GET

\section*{Структура запроса}
\begin{lstlisting}
localhost:9999/storage/{storageId}
\end{lstlisting}
\hfill

\section*{Параметры запроса}
\begin{table}[htbp]
    \centering
    \begin{tabularx}{\textwidth}{bsm}
    
    	\rowcolor{titleColor}
        \textbf{Параметр} & \textbf{Тип} & \textbf{Описание} \\  
        
         storageId & int  & Id пункта выдачи \\
    \end{tabularx}
\end{table}

\section*{Ответные данные}
\begin{table}[htbp]
    \centering
    \begin{tabularx}{\textwidth}{bsm}
    
    	\rowcolor{titleColor}
        \textbf{Параметр} & \textbf{Тип} & \textbf{Описание} \\  
        

		id & int  & Id пункта выдачи \\   \rowcolor{codeColor}
        coordinatesId & int  & Id координат пункта выдачи\\   
        sector & string  & Сектор \\ \rowcolor{codeColor}
        name & string  &  Наименование \\   
        phone & string  & Телефон \\ \rowcolor{codeColor}
    \end{tabularx}
\end{table}

\section*{Пример}

\subsection*{Запрос}

\begin{lstlisting}
localhost:9999/product/2
\end{lstlisting}
\hfill

\subsection*{Ответ}

\begin{lstlisting}
{
    "code": 200,
    "message": "OK",
    "data": {
        "id": 2,
        "typeId": 1,
        "name": "product name",
        "decription": "product description",
        "price": 100
    }
}
\end{lstlisting}
\hfill



\part{Таблица Coordinates}
%*****************************************************************************************************************************%
%*****************************************************************************************************************************%
%********************************************************Coordinates********************************************************%
%*****************************************************************************************************************************%
%*****************************************************************************************************************************%


%********************************************************getById********************************************************%
\chapter{getById}

\section*{Метод}
GET

\section*{Структура запроса}
\begin{lstlisting}
localhost:9999/coord/{coordId}
\end{lstlisting}
\hfill

\section*{Параметры запроса}
\begin{table}[htbp]
    \centering
    \begin{tabularx}{\textwidth}{bsm}
    
    	\rowcolor{titleColor}
        \textbf{Параметр} & \textbf{Тип} & \textbf{Описание} \\  
        
         coordId & int  & Id набора координат \\
    \end{tabularx}
\end{table}

\section*{Ответные данные}
\begin{table}[htbp]
    \centering
    \begin{tabularx}{\textwidth}{bsm}
    
    	\rowcolor{titleColor}
        \textbf{Параметр} & \textbf{Тип} & \textbf{Описание} \\  
        

		id & int  & Id набора координат \\   \rowcolor{codeColor}
        latitudeDeg & int  & Градусы широты \\   
        latitudeMin & int  & Минуты широты \\ \rowcolor{codeColor}
        latitudeSec & decimal  &  Секунды широты \\   
        longitudeDeg & int  & Градусы долготы \\ \rowcolor{codeColor}
        longitudeMin & int  & Минуты долготы \\ 
        longitudeSec & decimal  &  Секунды долготы \\ \rowcolor{codeColor}
    \end{tabularx}
\end{table}

\section*{Пример}

\subsection*{Запрос}

\begin{lstlisting}
localhost:9999/coord/2
\end{lstlisting}
\hfill

\subsection*{Ответ}

\begin{lstlisting}
{
    "code": 200,
    "message": "OK",
    "data": {
        "id": 2,
        "latitudeDeg": 1,
        "latitudeMin": 5
        "latitudeSec": 3.1
        "latitudeDeg": 5
        "latitudeMin": 8
        "latitudeSec": 11.5
    }
}
\end{lstlisting}
\hfill


%********************************************************getAll********************************************************%
\chapter{getAll}

\section*{Метод}
GET

\section*{Структура запроса}
\begin{lstlisting}
localhost:9999/allCoord
\end{lstlisting}
\hfill

\section*{Параметры запроса}
Отсутствуют.

\section*{Ответные данные}

\begin{table}[htbp]
    \centering
    \begin{tabularx}{\textwidth}{bsm}
    
    	\rowcolor{titleColor}
        \textbf{Параметр} & \textbf {Тип} & \textbf{Описание} \\  
        

        coordinates & list  & Список координат \\   \rowcolor{codeColor}

    \end{tabularx}
\end{table}

\section*{Пример}

\subsection*{Запрос}

\begin{lstlisting}
localhost:9999/allCoord
\end{lstlisting}
\hfill

\subsection*{Ответ}

\begin{lstlisting}
{
	"code" : 200,
	"message" : "OK" ,
	"data": [{
        "id": 2,
        "latitudeDeg": 1,
        "latitudeMin": 5
        "latitudeSec": 3.1
        "latitudeDeg": 5
        "latitudeMin": 8
        "latitudeSec": 11.5
	},
	{
        "id": 3,
        "latitudeDeg": 1,
        "latitudeMin": 5
        "latitudeSec": 10.1
        "latitudeDeg": 5
        "latitudeMin": 8
        "latitudeSec": 8.5
	}
	]
}
\end{lstlisting}
\hfill


%********************************************************delete********************************************************%
\chapter{delete}

\section*{Метод}
DELETE

\section*{Структура запроса}
\begin{lstlisting}
localhost:9999/coord/{coordId}
\end{lstlisting}
\hfill

\section*{Параметры запроса}
\begin{table}[htbp]
    \centering
    \begin{tabularx}{\textwidth}{bsm}
    
    	\rowcolor{titleColor}
        \textbf{Параметр} & \textbf {Тип} & \textbf{Описание} \\  
        
         coordId & int  & Id набора координат \\
    \end{tabularx}
\end{table}

\section*{Ответные данные}
Отсутствуют.

\section*{Пример}

\subsection*{Запрос}

\begin{lstlisting}
localhost:9999/coord/2
\end{lstlisting}
\hfill

\subsection*{Ответ}

\begin{lstlisting}
{
	"code": 200,
	"message": "OK"
}
\end{lstlisting}
\hfill


%********************************************************save********************************************************%
\chapter{save}

\section*{Метод}
POST

\section*{Структура запроса}
\begin{lstlisting}
localhost:9999/coord
\end{lstlisting}
\hfill

\section*{Параметры запроса}
\begin{table}[htbp]
    \centering
    \begin{tabularx}{\textwidth}{bsm}
    
    	\rowcolor{titleColor}
        \textbf{Параметр} & \textbf {Тип} & \textbf{Описание} \\  
        
        latitudeDeg & int  & Градусы широты \\   \rowcolor{codeColor}
        latitudeMin & int  & Минуты широты \\ 
        latitudeSec & decimal  &  Секунды широты \\   \rowcolor{codeColor}
        longitudeDeg & int  & Градусы долготы \\ 
        longitudeMin & int  & Минуты долготы \\ \rowcolor{codeColor}
        longitudeSec & decimal  &  Секунды долготы \\ 
    \end{tabularx}
\end{table}

\section*{Ответные данные}
Отсутствуют.

\section*{Пример}

\subsection*{Запрос}

\begin{lstlisting}
localhost:9999/coord
{
    "latitudeDeg": 1,
    "latitudeMin": 5
    "latitudeSec": 11.8
    "latitudeDeg": 5
    "latitudeMin": 8
    "latitudeSec": 5.4
}
\end{lstlisting}
\hfill

\subsection*{Ответ}

\begin{lstlisting}
{
	"code": 200,
	"message": "OK"
}
\end{lstlisting}
\hfill


%********************************************************edit********************************************************%
\chapter{edit}

\section*{Метод}
PATCH

\section*{Структура запроса}
\begin{lstlisting}
localhost:9999/coord
\end{lstlisting}
\hfill

\section*{Параметры запроса}
\begin{table}[htbp]
    \centering
    \begin{tabularx}{\textwidth}{bscm}
    
    	\rowcolor{titleColor}
        \textbf{Параметр} & \textbf {Тип} & \textbf {Обязательно} & \textbf{Описание} \\  
        
		id & int & Да & Id набора координат \\   \rowcolor{codeColor}
        latitudeDeg & int & Нет & Градусы широты \\   
        latitudeMin & int & Нет & Минуты широты \\ \rowcolor{codeColor}
        latitudeSec & decimal & Нет &  Секунды широты \\   
        longitudeDeg & int & Нет & Градусы долготы \\ \rowcolor{codeColor}
        longitudeMin & int & Нет & Минуты долготы \\ 
        longitudeSec & decimal & Нет &  Секунды долготы \\ \rowcolor{codeColor}
    \end{tabularx}
\end{table}

\section*{Ответные данные}
Отсутствуют.

\section*{Пример}

\subsection*{Запрос}

\begin{lstlisting}
localhost:9999/coord
{
	"id": 2,
	"latitudeSec": 5.1
}
\end{lstlisting}
\hfill

\subsection*{Ответ}

\begin{lstlisting}
{
	"code": 200,
	"message": "OK"
}
\end{lstlisting}
\hfill



\part{Таблица Employee}
%**************************************************************************************************************************%
%**************************************************************************************************************************%
%********************************************************Employee********************************************************%
%**************************************************************************************************************************%
%**************************************************************************************************************************%


%********************************************************getById********************************************************%
\chapter{getById}

\section*{Метод}
GET

\section*{Структура запроса}
\begin{lstlisting}
localhost:9999/employee/{employeeId}
\end{lstlisting}
\hfill

\section*{Параметры запроса}
\begin{table}[htbp]
    \centering
    \begin{tabularx}{\textwidth}{bsm}
    
    	\rowcolor{titleColor}
        \textbf{Параметр} & \textbf{Тип} & \textbf{Описание} \\  
        
         employeeId & int  & Id сотрудника \\
    \end{tabularx}
\end{table}

\section*{Ответные данные}
\begin{table}[htbp]
    \centering
    \begin{tabularx}{\textwidth}{bsm}
    
    	\rowcolor{titleColor}
        \textbf{Параметр} & \textbf{Тип} & \textbf{Описание} \\  
        

		id & int  & Id сотрудника \\   \rowcolor{codeColor}
        positionId & int  & Id должности \\   
        pasportId & int  & Id паспортных данных \\ \rowcolor{codeColor}
        name & string  &  Имя \\   
        surName & string  & Фамилия \\ \rowcolor{codeColor}
        oldName & string  & Отчество \\ 
        email & string  &  Электронный адрес \\ \rowcolor{codeColor}
        regDate & decimal  &  Дата регистрации \\   
        remDate & timestamp  & Дата удаления \\ \rowcolor{codeColor}
    \end{tabularx}
\end{table}

\section*{Пример}

\subsection*{Запрос}

\begin{lstlisting}
localhost:9999/employee/2
\end{lstlisting}
\hfill

\subsection*{Ответ}

\begin{lstlisting}
{
    "code": 200,
    "message": "OK",
    "data": {
        "id": 2,
        "positionId": 1,
        "pasportId": 2,
        "name": "Name",
        "surName": "Surname",
        "oldName": "Oldname",
        "email": "email",
        "regDate": 17.01.01,
        "remDate": 17.02.02
    }
}
\end{lstlisting}
\hfill

%********************************************************getAll********************************************************%
\chapter{getAll}

\section*{Метод}
GET

\section*{Структура запроса}
\begin{lstlisting}
localhost:9999/allEmployees
\end{lstlisting}
\hfill

\section*{Параметры запроса}
Отсутствуют.

\section*{Ответные данные}

\begin{table}[htbp]
    \centering
    \begin{tabularx}{\textwidth}{bsm}
    
    	\rowcolor{titleColor}
        \textbf{Параметр} & \textbf {Тип} & \textbf{Описание} \\  
        

        employees & list  & Список сотрудников \\   \rowcolor{codeColor}

    \end{tabularx}
\end{table}

\section*{Пример}

\subsection*{Запрос}

\begin{lstlisting}
localhost:9999/allEmployees
\end{lstlisting}
\hfill

\subsection*{Ответ}

\begin{lstlisting}
{
	"code" : 200,
	"message" : "OK" ,
	"data": [{
        "id": 2,
        "positionId": 1,
        "pasportId": 2,
        "name": "Name",
        "surName": "Surname",
        "oldName": "Oldname",
        "email": "email",
        "regDate": 17.01.01,
        "remDate": 17.02.02
	},
	{
        "id": 3,
        "positionId": 2,
        "pasportId": 3,
        "name": "NameOther",
        "surName": "SurnameOther",
        "oldName": "OldnameOther",
        "email": "emailOther",
        "regDate": 17.01.02,
        "remDate": 17.02.03
	}
	]
}
\end{lstlisting}
\hfill


%********************************************************delete********************************************************%
\chapter{delete}

\section*{Метод}
DELETE

\section*{Структура запроса}
\begin{lstlisting}
localhost:9999/employee/{employeeId}
\end{lstlisting}
\hfill

\section*{Параметры запроса}
\begin{table}[htbp]
    \centering
    \begin{tabularx}{\textwidth}{bsm}
    
    	\rowcolor{titleColor}
        \textbf{Параметр} & \textbf {Тип} & \textbf{Описание} \\  
        
         employeeId & int  & Id сотрудника \\
    \end{tabularx}
\end{table}

\section*{Ответные данные}
Отсутствуют.

\section*{Пример}

\subsection*{Запрос}

\begin{lstlisting}
localhost:9999/employee/2
\end{lstlisting}
\hfill

\subsection*{Ответ}

\begin{lstlisting}
{
	"code": 200,
	"message": "OK"
}
\end{lstlisting}
\hfill


%********************************************************save********************************************************%
\chapter{save}

\section*{Метод}
POST

\section*{Структура запроса}
\begin{lstlisting}
localhost:9999/employee
\end{lstlisting}
\hfill

\section*{Параметры запроса}
\begin{table}[htbp]
    \centering
    \begin{tabularx}{\textwidth}{bsm}
    
    	\rowcolor{titleColor}
        \textbf{Параметр} & \textbf {Тип} & \textbf{Описание} \\  
        
        positionId & int  & Id должности \\   \rowcolor{codeColor}
        pasportId & int  & Id паспортных данных \\ 
        name & string  &  Имя \\   \rowcolor{codeColor}
        surName & string  & Фамилия \\ 
        oldName & string  & Отчество \\ \rowcolor{codeColor}
        email & string  &  Электронный адрес \\
        regDate & decimal  &  Дата регистрации \\   \rowcolor{codeColor}
        remDate & timestamp  & Дата удаления \\ 
    \end{tabularx}
\end{table}

\section*{Ответные данные}
Отсутствуют.

\section*{Пример}

\subsection*{Запрос}

\begin{lstlisting}
localhost:9999/employee
{
	"positionId": 2,
	"pasportId": 3,
	"name": "someName",
	"surName "someSurname",
	"oldName": "someOldname",
	"email ": "someEmail",
	"regDate ": 17.02.02,
	"remDate": 17.03.03
}
\end{lstlisting}
\hfill

\subsection*{Ответ}

\begin{lstlisting}
{
	"code": 200,
	"message": "OK"
}
\end{lstlisting}
\hfill


%********************************************************edit********************************************************%
\chapter{edit}

\section*{Метод}
PATCH

\section*{Структура запроса}
\begin{lstlisting}
localhost:9999/employee
\end{lstlisting}
\hfill

\section*{Параметры запроса}
\begin{table}[htbp]
    \centering
    \begin{tabularx}{\textwidth}{bscm}
    
    	\rowcolor{titleColor}
        \textbf{Параметр} & \textbf {Тип} & \textbf {Обязательно} & \textbf{Описание} \\  
        
		id & int & Да & Id сотрудника \\   \rowcolor{codeColor}
        positionId & int & Нет & Id должности \\   
        pasportId & int & Нет & Id паспортных данных \\ \rowcolor{codeColor}
        name & string & Нет & Имя \\   
        surName & string & Нет & Фамилия \\ \rowcolor{codeColor}
        oldName & string & Нет & Отчество \\ 
        email & string & Нет & Электронный адрес \\ \rowcolor{codeColor}
        regDate & decimal & Нет & Дата регистрации \\   
        remDate & timestamp & Нет & Дата удаления \\ \rowcolor{codeColor}
    \end{tabularx}
\end{table}

\section*{Ответные данные}
Отсутствуют.

\section*{Пример}

\subsection*{Запрос}

\begin{lstlisting}
localhost:9999/employee
{
	"id": 2,
	"surName": "otherSurName"
}
\end{lstlisting}
\hfill

\subsection*{Ответ}

\begin{lstlisting}
{
	"code": 200,
	"message": "OK"
}
\end{lstlisting}
\hfill



\part{Таблица EmployeeWorks}


\end{document}