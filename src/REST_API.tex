\documentclass[14pt,a4paper,report]{report}
\usepackage[a4paper, mag=1000, left=2.5cm, right=1cm, top=2cm, bottom=2cm, headsep=0.7cm, footskip=1cm]{geometry}
\usepackage[utf8]{inputenc}
\usepackage[english,russian]{babel}
\usepackage{indentfirst}
\usepackage[dvipsnames]{xcolor}
\usepackage[colorlinks]{hyperref}
\usepackage{listings} 
\usepackage{fancyhdr}
\usepackage{caption}
\usepackage{graphicx}
\hypersetup{
	colorlinks = true,
	linkcolor  = black
}

\usepackage{titlesec}
\titleformat{\chapter}
{\Large\bfseries} % format
{}                % label
{0pt}             % sep
{\huge}           % before-code

\definecolor{codeColor}{RGB}{246, 248, 250}
\definecolor{titleColor}{RGB}{240, 237, 252}

\DeclareCaptionFont{white}{\color{white}} 

% Listing description
\usepackage{listings} 
\DeclareCaptionFormat{listing}{\colorbox{gray}{\parbox{\textwidth}{#1#2#3}}}
\captionsetup[lstlisting]{format=listing,labelfont=white,textfont=white}
\lstset{ 
	% Listing settings
	inputencoding = utf8,			
	extendedchars = \true, 
	keepspaces = true, 			  	 % Поддержка кириллицы и пробелов в комментариях
	language = Java,            	 	 % Язык программирования (для подсветки)
	stepnumber = 1,               	 % Размер шага между двумя номерами строк
	numbersep = 5pt,              	 % Как далеко отстоят номера строк от подсвечиваемого кода
	backgroundcolor = \color{codeColor}, % Цвет фона подсветки - используем \usepackage{color}
	showspaces = false,           	 % Показывать или нет пробелы специальными отступами
	showstringspaces = false,    	 % Показывать или нет пробелы в строках
	showtabs = false,           	 % Показывать или нет табуляцию в строках
	tabsize = 2,                  	 % Размер табуляции по умолчанию равен 2 пробелам
	captionpos = t,             	 % Позиция заголовка вверху [t] или внизу [b] 
	breaklines = true,           	 % Автоматически переносить строки (да\нет)
	breakatwhitespace = false   	 % Переносить строки только если есть пробел
}

\usepackage{tabularx}
 \newcolumntype{b}{>{\hsize=0.2\hsize}X}
\newcolumntype{s}{>{\hsize=0.15\hsize}X}
\newcolumntype{m}{>{\hsize=1\hsize}X}
\usepackage{color, colortbl}

\begin{document}

\begin{titlepage}
\vspace*{\fill}
    \begin{center}
      \textbf{\Huge RESTful API}

    \end{center}
    \vspace*{\fill}
\end{titlepage}

\setcounter{page}{2}

\def\contentsname{Содержание}
\tableofcontents
\clearpage

\part{Общая информация}

\section*{Заголовок запроса}
\begin{lstlisting}
[{"key":"Content-Type","value":"application/json"}]
\end{lstlisting}
\hfill

\section*{Ответные данные}       
    \begin{table}[htbp]
    \centering
    \begin{tabularx}{\textwidth}{bsm}
    
    	\rowcolor{titleColor}
        \textbf{Параметр} & \textbf {Тип} & \textbf{Описание} \\  
        
        code & int  & Код ответа \\    \rowcolor{codeColor}
        message & string  & Описание кода ответа \\
        data & struct & Данные \\
    \end{tabularx}
\end{table}

\section*{Коды ответных данных}
\begin{table}[htbp]
    \centering
    \begin{tabularx}{\textwidth}{bm}
    
    	\rowcolor{titleColor}
    	\textbf{Код} & \textbf{Описание} \\  
        
        200 & Запрос выполнен успешно \\   \rowcolor{codeColor}
        400 & Запрос не удалось обработать из-за синтаксической ошибки \\
        404 & Сервер не нашел ресурсов \\   \rowcolor{codeColor}
        500 & Сервер не смог обработать запрос \\
    \end{tabularx}
\end{table}

\part{Таблица Client}

\chapter{getById}

\section*{Метод}
GET

\section*{Структура запроса}
\begin{lstlisting}
localhost:9999/client/{userId}
\end{lstlisting}
\hfill

\section*{Параметры запроса}
\begin{table}[htbp]
    \centering
    \begin{tabularx}{\textwidth}{bsm}
    
    	\rowcolor{titleColor}
        \textbf{Параметр} & \textbf {Тип} & \textbf{Описание} \\  
        
         userId & int  & Id клиента \\
    \end{tabularx}
\end{table}

\section*{Ответные данные}

\begin{table}[htbp]
    \centering
    \begin{tabularx}{\textwidth}{bsm}
    
    	\rowcolor{titleColor}
        \textbf{Параметр} & \textbf {Тип} & \textbf{Описание} \\  
        

        buys & list  & История покупок \\   \rowcolor{codeColor}
        name & string  & Имя \\   
        surName & string  & Фамилия \\ \rowcolor{codeColor}
        oldName & string  &  Отчество \\   
        email & string  & Электронный адрес \\ \rowcolor{codeColor}
        regDate & datetime  & Дата регистрации учетной записи \\ 
        remDate & datetime  & Дата удаления учетной записи  \\ \rowcolor{codeColor}
        id & int  & Id клиента \\  
        idValidate & bool  & Признак подтверждения учетной записи \\  \rowcolor{codeColor}
    \end{tabularx}
\end{table}

\section*{Пример}

\subsection*{Запрос}

\begin{lstlisting}
localhost:9999/client/2
\end{lstlisting}
\hfill

\subsection*{Ответ}

\begin{lstlisting}
{
    "code": 200,
    "message": "OK",
    "data": {
        "buys": [],
        "surName": null,
        "oldName": null,
        "name": "denis",
        "regDate": null,
        "remDate": null,
        "id": 2,
        "email": "myEmail"
    }
}
\end{lstlisting}
\hfill


\chapter{getAll}

\section*{Метод}
GET

\section*{Структура запроса}
\begin{lstlisting}
localhost:9999/client/{userId}
\end{lstlisting}
\hfill

\section*{Параметры запроса}
\begin{table}[htbp]
    \centering
    \begin{tabularx}{\textwidth}{bsm}
    
    	\rowcolor{titleColor}
        \textbf{Параметр} & \textbf {Тип} & \textbf{Описание} \\  
        
         userId & int  & Id клиента \\
    \end{tabularx}
\end{table}

\section*{Ответные данные}

\begin{table}[htbp]
    \centering
    \begin{tabularx}{\textwidth}{bsm}
    
    	\rowcolor{titleColor}
        \textbf{Параметр} & \textbf {Тип} & \textbf{Описание} \\  
        

        buys & list  & История покупок \\   \rowcolor{codeColor}
        name & string  & Имя \\   
        surName & string  & Фамилия \\ \rowcolor{codeColor}
        oldName & string  &  Отчество \\   
        email & string  & Электронный адрес \\ \rowcolor{codeColor}
        regDate & datetime  & Дата регистрации учетной записи \\ 
        remDate & datetime  & Дата удаления учетной записи  \\ \rowcolor{codeColor}
        id & int  & Id клиента \\  
        idValidate & bool  & Признак подтверждения учетной записи \\  \rowcolor{codeColor}
    \end{tabularx}
\end{table}

\section*{Пример}

\subsection*{Запрос}

\begin{lstlisting}
localhost:9999/client/2
\end{lstlisting}
\hfill

\subsection*{Ответ}

\begin{lstlisting}
{
    "code": 200,
    "message": "OK",
    "data": {
        "buys": [],
        "surName": null,
        "oldName": null,
        "name": "denis",
        "regDate": null,
        "remDate": null,
        "id": 2,
        "email": "myEmail"
    }
}
\end{lstlisting}
\hfill

\chapter{delete}

\section*{Метод}
DELETE

\section*{Структура запроса}
\begin{lstlisting}
localhost:9999/client/{userId}
\end{lstlisting}
\hfill

\section*{Параметры запроса}
\begin{table}[htbp]
    \centering
    \begin{tabularx}{\textwidth}{bsm}
    
    	\rowcolor{titleColor}
        \textbf{Параметр} & \textbf {Тип} & \textbf{Описание} \\  
        
         userId & int  & Id клиента \\
    \end{tabularx}
\end{table}

\section*{Ответные данные}

\begin{table}[htbp]
    \centering
    \begin{tabularx}{\textwidth}{bsm}
    
    	\rowcolor{titleColor}
        \textbf{Параметр} & \textbf {Тип} & \textbf{Описание} \\  
        

        buys & list  & История покупок \\   \rowcolor{codeColor}
        name & string  & Имя \\   
        surName & string  & Фамилия \\ \rowcolor{codeColor}
        oldName & string  &  Отчество \\   
        email & string  & Электронный адрес \\ \rowcolor{codeColor}
        regDate & datetime  & Дата регистрации учетной записи \\ 
        remDate & datetime  & Дата удаления учетной записи  \\ \rowcolor{codeColor}
        id & int  & Id клиента \\  
        idValidate & bool  & Признак подтверждения учетной записи \\  \rowcolor{codeColor}
    \end{tabularx}
\end{table}

\section*{Пример}

\subsection*{Запрос}

\begin{lstlisting}
localhost:9999/client/2
\end{lstlisting}
\hfill

\subsection*{Ответ}

\begin{lstlisting}
{
    "code": 200,
    "message": "OK",
    "data": {
        "buys": [],
        "surName": null,
        "oldName": null,
        "name": "denis",
        "regDate": null,
        "remDate": null,
        "id": 2,
        "email": "myEmail"
    }
}
\end{lstlisting}
\hfill


\chapter{save}

\section*{Метод}
POST

\section*{Структура запроса}
\begin{lstlisting}
localhost:9999/client/{userId}
\end{lstlisting}
\hfill

\section*{Параметры запроса}
\begin{table}[htbp]
    \centering
    \begin{tabularx}{\textwidth}{bsm}
    
    	\rowcolor{titleColor}
        \textbf{Параметр} & \textbf {Тип} & \textbf{Описание} \\  
        
         userId & int  & Id клиента \\
    \end{tabularx}
\end{table}

\section*{Ответные данные}

\begin{table}[htbp]
    \centering
    \begin{tabularx}{\textwidth}{bsm}
    
    	\rowcolor{titleColor}
        \textbf{Параметр} & \textbf {Тип} & \textbf{Описание} \\  
        

        buys & list  & История покупок \\   \rowcolor{codeColor}
        name & string  & Имя \\   
        surName & string  & Фамилия \\ \rowcolor{codeColor}
        oldName & string  &  Отчество \\   
        email & string  & Электронный адрес \\ \rowcolor{codeColor}
        regDate & datetime  & Дата регистрации учетной записи \\ 
        remDate & datetime  & Дата удаления учетной записи  \\ \rowcolor{codeColor}
        id & int  & Id клиента \\  
        idValidate & bool  & Признак подтверждения учетной записи \\  \rowcolor{codeColor}
    \end{tabularx}
\end{table}

\section*{Пример}

\subsection*{Запрос}

\begin{lstlisting}
localhost:9999/client/2
\end{lstlisting}
\hfill

\subsection*{Ответ}

\begin{lstlisting}
{
    "code": 200,
    "message": "OK",
    "data": {
        "buys": [],
        "surName": null,
        "oldName": null,
        "name": "denis",
        "regDate": null,
        "remDate": null,
        "id": 2,
        "email": "myEmail"
    }
}
\end{lstlisting}
\hfill


\chapter{edit}

\section*{Метод}
PATCH

\section*{Структура запроса}
\begin{lstlisting}
localhost:9999/client/{userId}
\end{lstlisting}
\hfill

\section*{Параметры запроса}
\begin{table}[htbp]
    \centering
    \begin{tabularx}{\textwidth}{bsm}
    
    	\rowcolor{titleColor}
        \textbf{Параметр} & \textbf {Тип} & \textbf{Описание} \\  
        
         userId & int  & Id клиента \\
    \end{tabularx}
\end{table}

\section*{Ответные данные}

\begin{table}[htbp]
    \centering
    \begin{tabularx}{\textwidth}{bsm}
    
    	\rowcolor{titleColor}
        \textbf{Параметр} & \textbf {Тип} & \textbf{Описание} \\  
        

        buys & list  & История покупок \\   \rowcolor{codeColor}
        name & string  & Имя \\   
        surName & string  & Фамилия \\ \rowcolor{codeColor}
        oldName & string  &  Отчество \\   
        email & string  & Электронный адрес \\ \rowcolor{codeColor}
        regDate & datetime  & Дата регистрации учетной записи \\ 
        remDate & datetime  & Дата удаления учетной записи  \\ \rowcolor{codeColor}
        id & int  & Id клиента \\  
        idValidate & bool  & Признак подтверждения учетной записи \\  \rowcolor{codeColor}
    \end{tabularx}
\end{table}

\section*{Пример}

\subsection*{Запрос}

\begin{lstlisting}
localhost:9999/client/2
\end{lstlisting}
\hfill

\subsection*{Ответ}

\begin{lstlisting}
{
    "code": 200,
    "message": "OK",
    "data": {
        "buys": [],
        "surName": null,
        "oldName": null,
        "name": "denis",
        "regDate": null,
        "remDate": null,
        "id": 2,
        "email": "myEmail"
    }
}
\end{lstlisting}
\hfill


\part{Таблица SellEntre}
\part{Таблица Product}
\part{Таблица Type}
\part{Таблица Sell}
\part{Таблица ProductAvaliability}
\part{Таблица Storage}
\part{Таблица Coordinates}
\part{Таблица Employee}
\part{Таблица EmployeeWorks}



\end{document}