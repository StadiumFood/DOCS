\documentclass[14pt,a4paper,report]{report}
\usepackage[a4paper, mag=1000, left=2.5cm, right=1cm, top=2cm, bottom=2cm, headsep=0.7cm, footskip=1cm]{geometry}
\usepackage[utf8]{inputenc}
\usepackage[english,russian]{babel}
\usepackage{indentfirst}
\usepackage[dvipsnames]{xcolor}
\usepackage[colorlinks]{hyperref}
\usepackage{listings} 
\usepackage{fancyhdr}
\usepackage{caption}
\usepackage{graphicx}
\hypersetup{
	colorlinks = true,
	linkcolor  = black
}

\usepackage{titlesec}
\titleformat{\chapter}
{\Large\bfseries} % format
{}                % label
{0pt}             % sep
{\huge}           % before-code

\definecolor{codeColor}{RGB}{246, 248, 250}
\definecolor{titleColor}{RGB}{240, 237, 252}

\DeclareCaptionFont{white}{\color{white}} 

% Listing description
\usepackage{listings} 
\DeclareCaptionFormat{listing}{\colorbox{gray}{\parbox{\textwidth}{#1#2#3}}}
\captionsetup[lstlisting]{format=listing,labelfont=white,textfont=white}
\lstset{ 
	% Listing settings
	inputencoding = utf8,			
	extendedchars = \true, 
	keepspaces = true, 			  	 % Поддержка кириллицы и пробелов в комментариях
	language = Java,            	 	 % Язык программирования (для подсветки)
	stepnumber = 1,               	 % Размер шага между двумя номерами строк
	numbersep = 5pt,              	 % Как далеко отстоят номера строк от подсвечиваемого кода
	backgroundcolor = \color{codeColor}, % Цвет фона подсветки - используем \usepackage{color}
	showspaces = false,           	 % Показывать или нет пробелы специальными отступами
	showstringspaces = false,    	 % Показывать или нет пробелы в строках
	showtabs = false,           	 % Показывать или нет табуляцию в строках
	tabsize = 2,                  	 % Размер табуляции по умолчанию равен 2 пробелам
	captionpos = t,             	 % Позиция заголовка вверху [t] или внизу [b] 
	breaklines = true,           	 % Автоматически переносить строки (да\нет)
	breakatwhitespace = false   	 % Переносить строки только если есть пробел
}

\usepackage{tabularx}
 \newcolumntype{b}{>{\hsize=0.2\hsize}X}
\newcolumntype{s}{>{\hsize=0.15\hsize}X}
\newcolumntype{m}{>{\hsize=1\hsize}X}
\usepackage{color, colortbl}

\begin{document}


\begin{titlepage}
\vspace*{\fill}
    \begin{center}
      \textbf{\Huge RESTful API}

    \end{center}
    \vspace*{\fill}
\end{titlepage}


\def\contentsname{Содержание}
\tableofcontents
\clearpage


\part{Таблица Client}

\chapter{getById}

\section*{Метод}
GET

\section*{Параметры}
GET

\section*{Ответные данные}


\begin{table}[htbp]
    \centering
    \begin{tabularx}{\textwidth}{bsm}
    
    	\rowcolor{titleColor}
        Параметр & Тип & Описание \\  
        
        id & int  & Код ответа \\
        \rowcolor{codeColor}
        id & int  & Код ответа \\
    \end{tabularx}
\end{table}





\section*{Коды ответов}

\section*{Пример}

\subsection*{Запрос}
\subsection*{Ответ}

\begin{lstlisting}
	$.ajax({
    	url: "/users/1",
    	dataType: "json",
    	type : "GET",
    	success : function(r) {
      	console.log(r);
    	}
  	});
\end{lstlisting}


\chapter{getAll}



\chapter{delete}



\chapter{save}


\chapter{edit}



\part{Таблица SellEntre}
\part{Таблица Product}
\part{Таблица Type}
\part{Таблица Sell}
\part{Таблица ProductAvaliability}
\part{Таблица Storage}
\part{Таблица Stadium}



\end{document}